\documentclass{article}[11pt]
\usepackage[left=1.0in, right=1.0in, top=1.0in, bottom=1.0in,nohead]{geometry}              		
\geometry{letterpaper}

\usepackage{graphicx, grffile}														
\usepackage{amssymb, amsmath, amsfonts}
\usepackage{setspace}
\usepackage{titlesec}
\usepackage[usenames, dvipsnames]{color}
\usepackage{soul}
\usepackage{float, multirow, tabulary}
\usepackage{subcaption}

\newcommand{\bp}[1] {\left( #1 \right)}
\newcommand{\bb}[1] {\left[ #1 \right]}
\newcommand{\ba}[1]{\left| #1 \right|}
\newcommand{\infint}{\int_{-\infty}^\infty}

\titleformat{\section}{\Large\bfseries}{\thesection}{0.5em}{}
  
\titleformat{\subsection}{\normalfont\normalsize\bfseries}{\thesubsection}{0.5em}{}

\titleformat{\subsubsection}{\normalfont\normalsize\bfseries\itshape}{\thesubsubsection}{0.5em}{}  
  
\titlespacing{\section}{0pt}{0pt}{2pt}

%\newcommand{\kbcom}[1]{
%\textcolor{blue}{\textbf{\textit{\ul{#1}}}}}
%
%\newcommand{\blcom}[1]{
%\textcolor{red}{\textbf{\textit{\ul{#1}}}}}         
%
%\newcommand{\llcom}[1]{
%\textcolor{Orange}{\textbf{\textit{\ul{#1}}}}}
%
%\newcommand{\tlcom}[1]{
%\textcolor{LimeGreen}{\textbf{\textit{\ul{#1}}}}} 

\newcommand{\Matlab}{\textsc{Matlab}}

\title{Final Report on Parameter Estimation for Nonlinear Dynamical Systems}
\author{Brett Larsen and Ajay Karpur}
\date{}


%%%

\begin{document}
\maketitle
\doublespace

%\vspace{-30pt}

\begin{abstract}
The purpose of this project was to explore the methods by which one could estimate the values of parameters underlying a nonlinear dynamical system. To accomplish this, we examined the parameter $\beta$ underlying a simulation of the Lorenz system in \Matlab.
\end{abstract}

\section{Introduction}
\label{sec:intro}
In estimation theory, parameter estimation is used to determine approximate values of parameters from a set of measured data that have some degree of randomness. By estimating the value of these parameters, one can approximate the conditions underlying some system. Parameter estimation has applications in a variety of engineering problems. When mathematical models are used to describe biological or other nonlinear dynamical phenomena, these models often contain some parameters that cannot be directly quantified or calculated. In these cases, the parameters can be estimated using the available data.


%%%%%%%%%%%%%%%%%%%%%%%%%%%%%%%%%%%%%%%%%%%%%%%%%%%%%%
%\vspace{-10pt}
%\section{Background}
%\label{sec:background}



%%%%%%%%%%%%%%%%%%%%%%%%%%%%%%%%%%%%%%%%%%%%%%%%%%%%%%
%\vspace{-10pt}
\section{Methods}
\label{sec:methods}

To acquire random data, we used a \Matlab \ simulation of the Lorenz system.


%%%%%%%%%%%%%%%%%%%%%%%%%%%%%%%%%%%%%%%%%%%%%%%%%%%%%%
%\vspace{-10pt}
\section{Results}
\label{sec:results}

 

%\vspace{-10pt}
\section{Conclusions}
\label{sec:conclusion}


\section{Acknowledgements}
\label{sec:Ack}
We would like to acknowledge Moiseev Igor for the open source \Matlab \ Lorenz system generator.


%Bibliography
\bibliographystyle{IEEEtran}
\bibliography{proposal_refs.bib}

\end{document}